\documentclass[10.5pt][UTF8]{ctexart}\usepackage{caption}\usepackage{graphicx, subfig}\usepackage[UTF8]{ctex}\begin{document}

11. 在体育课的跳远比赛中,以${4} . {0} {0}$米为标准,若小东跳出了4.22米,可记做$+ {0} . {2} {2}$,那么小东跳出了3.85米,记作\_\_\_\_\_\_\_\_\_\_\_\_米.

12. 用四舍五入法将1.950取近似数并精确到十分位,得到的值是\_\_\_\_\_\_\_\_\_\_\_\_$| {a} + {1} | + | {a} |$

13. 若$| {a} - {2} | + \left( {b} + {3} \right) ^ {2} = {0}$则${b} ^ {a}$等于\_\_\_\_\_\_\_\_\_\_\_\_.

\includegraphics[width=71.69045005488475mm]{/home/lsk/Desktop/Gaosi/formula_19_0212/data/test111/122/旧/6_0_img/572_77_883_319.jpg}



14. 右图是一所住宅的建筑平面图(图中长度单位:$m )$,

这所住宅的建筑面积为\_\_\_\_\_\_\_\_\_\_\_\_\ \ \ \ \_\_\_\_\_\_${ { \text{m} } ^ {2} }$.

15. 写出一个满足“未知数的系数是$- {2}$,方程的解为${3} ^ {3}$

的一元一次方程:

\_\_\_\_\_\_\ \ \ \ \_\_\_\_\_\_.

16. 下列式子${x} ^ {2} + {2} , \frac { {1} } { {a} } + {4} , \frac { {3} {a} {b} ^ {2} } { {7} } , \frac { {a} {b} - {c} } { \pi } , - {5} {x} , {0}$中,整式有\_\_\_\_\_\_个.

18. 观察下列算式,你发现了什么规律?

${1} ^ {2} = \frac { {1} \times {2} \times {3} } { {6} }$${1} ^ {2} + {2} ^ {2} = \frac { {2} \times {3} \times {5} } { {6} }$${1} ^ {2} + {2} ^ {2} + {3} ^ {2} = \frac { {3} \times {4} \times {7} } { {6} }$

${1} ^ {2} + {2} ^ {2} + {3} ^ {2} + {4} ^ {2} = \frac { {4} \times {5} \times {9} } { {6} }$$\cdots$

①根据你发现的规律,计算下面算式的值;${1} ^ {2} + {2} ^ {2} + {3} ^ {2} + {4} ^ {2} + {5} ^ {2} =$\_\_\_\_\_\_\_\_\_\_\_\_.

②请用一个含n的算式表示这个规律:${1} ^ {2} + {2} ^ {2} + {3} ^ {2} \dots + {n} ^ {2} =$\_\_\_\_\_\_\_\_\_\_\_\_.

\end{document}